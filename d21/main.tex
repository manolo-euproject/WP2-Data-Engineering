\documentclass{report}

\usepackage{natbib}
\usepackage{graphicx}
\graphicspath{{figures/}}


\newcommand{\term}[1]{\textit{#1}}
\newcommand{\quotes}[1]{`#1'}

\newcommand{\mycomment}[2]
{
\begin{quote}
\textbf{#1:}
     #2
\end{quote}
}
%\renewcommand{\mycomment}[2]{}

\title{Data Inspection and Generation v.1}

\begin{document}

\maketitle

\chapter{Introduction}

\section{Scope of Deliverable}

This report, titled \quotes{Data Inspection and Generation v.1},
documents research \& development work carried out in WP2 during
Phase~2 \emph{MANOLO Framework Implementation} of the project's
workplan.

Besides the report itself, the scope of this deliverable also comprises
the intermediate versions of the following software components:
%
\begin{itemize}
\item The \emph{Data Operations Manager}
\item The \emph{Data Quality Estimation Component}
\item The \emph{Data Distillation and Synthesis Component}
\end{itemize}

Work in WP2 also includes preparing and ingesting the use case
datasets in the data operations manager, but this is outside the scope
of this version of the deliverable and will be reported in v.2 (M24).

\section{Structure of Deliverable}

Taking the above into consideration, this remainder of this document
is structured as follows:
%
\begin{itemize}
\item Chapter 2 -- TITLE: MANOLO presents to its cloud-edge operators
  a complicated provenance and lineage environment where the different
  assets, i.e., datasets, algorithms, models, resources are multiply
  interlinked, providing explanation of the capacities, data, metadata
  and their relationships. For regulatory as well as pragmatic
  reasons, MANOLO will need to ensure that project assets are
  integrated with the MANOLO data (assets) management sub-system, so
  that provenance and lineage metadata is automatically maintained.
  The Data Operations Manager will manage this functionality and
  expose APIs which will be supporting the overall project and MANOLO
  research and toolset.
\item Chapter 3 -- TITLE: Mechanisms for data quality estimation will
  be developed, tested and integrated in this component by focussing
  on detecting and correcting anomalous data and automatically
  annotating data in terms of quality. Mechanisms for noise detection
  (including biased data due to gender, race, or other variables) as
  well as data maliciously manipulated will be given a special
  emphasis, employing adversarial machine learning while considering
  associated models.
\item Chapter 4 -- TITLE: Techniques for data distillation will be
  explored here such that will provide a good foundation and a richer
  dataset to support the research in the Hardware-aware Model Training
  and Optimisation component. This will produce new derived
  (synthetic) data using methods for distillation via data compression
  and hashing, feature extraction and synthesisation; and model
  inversion for synthesisation of data from labels. Data compression
  will ensure the reduction of storage necessities as well as a
  faster, while accurate, training pipelines and lighter
  models. Feature extraction and synthesisation will allow us to
  propose meta-data for meta learning tasks. The creation of synthetic
  data from labels is a technique that will help us gather reliable
  datasets from accurate pretrained architectures to keep a useful
  repository for their application to posterior training processes
  where data are not satisfactory for quality, quantity or
  availability due to ethical reasons.
\end{itemize}

\chapter{Data Management and Provenance Framework}

% (M4-M12, M21-M24) (Leader: ARX.NET S.A., support:
%NUIDUCD-CeADAR, NCSR "D", FDI, ARCADA, PAL ROBOTICS, Bit&Brain)

\mycomment{S}{Task description}{%
MANOLO's Data Inspection \& Generation component/framework will be
implemented in this task as a generic cloud-based repository in which
all Digital Artefacts of the project that are used across the
edge-cloud continuum will be registered and stored. The artefacts
could be any type of data sample, datasets, metadata, AI models,
benchmark configuration and results, usage analytics and performance
metrics which can be transferred between any requesting component of
MANOLO through the use of an appropriate API. The API will provide all
the necessary end points for user authentication implementing a secure
identification scheme, for registering the artefacts and for
manipulating them via the relevant operations execution. All data
transfers from the cloud repository to any other location of the
edge-cloud continuum and vice versa, will be encrypted and
concurrently a low latency application layer network protocol will be
employed. This architecture provides a trustworthy solution for
general use, with the greatest benefits expected to arise from the
deployment on the Edge devices. A provenance and lineage system will
be developed to relate both the data, models and metadata both
inputted and derived from the application of different techniques of
MANOLO which will support the overall activities of MANOLO in terms of
algorithm design, testing, benchmarking and validation enhancing the
explainability and reproducibility. The repository will be enhanced
with metadata about (a) data quality (see T2.2); and the auxiliary
data and meta-data derived from distillation and synthetic
functionality (see T2.3). As the repository will expose an API through
which all data and model manipulation tools developed in MANOLO will
be able to register the operations applied, metadata is automatically
maintained through the usage of the tools without relying on explicit
metadata-related actions by the users. This task will also apply the
methods developed in T2.2 and T2.3 to populate the repository with the
data and models needed for the pilots.}


\chapter{Data Quality Estimation}
% (M5-M24) Leader: NCSR "D"
% CeADAR, UPC, ATOS, EVIDEN, FDI, INRIA, ARCADA, use case partners

\section{Introduction}

This task will implement a range of methods for automatically
annotating data with respect to quality or possible unreliability.

We will explore anomaly detection and other noise detection methods
that will allow identifying biased, noisy, inconsistent, or in general
low-quality data, as well as data maliciously manipulated to
contaminate the model during training. For these purposes, data
augmentation techniques based on adversarial machine learning methods
(such as controlled noise contamination of the dataset) will be used.

\section{Related Work}

Anomaly detection and other prior works that are domain-agnostic or
easily transferrable.


\section{Directing Attention at Low-Quality Data}

Background on Attention and the Transformer.

Core idea: observing the trained Attention module can identify the
characteristics of signal where attention was \emph{not} directed.
The expectation is that (ideally) loss due to inconsistent data will
be handled by directing attention away from inconsistent data instead
of training the network to work around it.


\section{Data Augmentation and Adversarial ML}

\mycomment{Expected contributions}%
{CeADAR to contribute regarding data augmentation}


\section{Experiments on Timeseries Processing}

\mycomment{Expected contributions}%
{BitBrain to contribute regarding background work on EEG quality estimation.}

\mycomment{Expected contributions}%
{PAL and BitBrain to contribute regarding the experimental setup.}

\section{Experiments on Computer Vision}

\mycomment{Expected contributions}%
{ARX.NET to contribute regarding background work on image quality estimation.}

\mycomment{Expected contributions}%
{PAL and ARX.NET to contribute regarding the experimental setup.}


\section{Conclusions}



\chapter{Data Distillation and Synthetic Feature Representation}
%(M5-M24) (Leader: NUIDUCD-CeADAR, support: NCSR
%"D", FDI, Fraunhofer, INRIA)

\mycomment{S}{Task description}{%
The goal of this task is to create a framework that incorporates
techniques aimed to provide distilled high quality data and meta-data
in cases when it is not available for the training phase of AI models
and when, if available, using the original data for training implies
the choice of an architecture that does not suit our objectives on
efficiency. To achieve this, we will develop methods for data
distillation, data compression and hashing, feature extraction and
synthesisation, as well as model inversion for synthesisation of data
from labels. This will be tested on the datasets collected during the
project and thus will be integrated in the framework of the Data
Inspection and Generation component developed in Task 2.1. The
outcomes of this task will be key to feed WP3 with data and meta-data
to support the development of novel algorithms.}


\chapter{Conclusion}

\end{document}
