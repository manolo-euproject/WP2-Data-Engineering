
% (M4-M12, M21-M24) (Leader: ARX.NET S.A., support:
%NUIDUCD-CeADAR, NCSR "D", FDI, ARCADA, PAL ROBOTICS, Bit&Brain)

\mycomment{S}{Task description}{%
MANOLO's Data Inspection \& Generation component/framework will be
implemented in this task as a generic cloud-based repository in which
all Digital Artefacts of the project that are used across the
edge-cloud continuum will be registered and stored. The artefacts
could be any type of data sample, datasets, metadata, AI models,
benchmark configuration and results, usage analytics and performance
metrics which can be transferred between any requesting component of
MANOLO through the use of an appropriate API. The API will provide all
the necessary end points for user authentication implementing a secure
identification scheme, for registering the artefacts and for
manipulating them via the relevant operations execution. All data
transfers from the cloud repository to any other location of the
edge-cloud continuum and vice versa, will be encrypted and
concurrently a low latency application layer network protocol will be
employed. This architecture provides a trustworthy solution for
general use, with the greatest benefits expected to arise from the
deployment on the Edge devices. A provenance and lineage system will
be developed to relate both the data, models and metadata both
inputted and derived from the application of different techniques of
MANOLO which will support the overall activities of MANOLO in terms of
algorithm design, testing, benchmarking and validation enhancing the
explainability and reproducibility. The repository will be enhanced
with metadata about (a) data quality (see T2.2); and the auxiliary
data and meta-data derived from distillation and synthetic
functionality (see T2.3). As the repository will expose an API through
which all data and model manipulation tools developed in MANOLO will
be able to register the operations applied, metadata is automatically
maintained through the usage of the tools without relying on explicit
metadata-related actions by the users. This task will also apply the
methods developed in T2.2 and T2.3 to populate the repository with the
data and models needed for the pilots.}
